% These are the lecture notes for my CSCI360 course SPRING 2017
% at John Jay College of Criminal Justice.

% Feel free to edit these slides and use them for your own courses.
% HOWEVER DO NOT REMOVE THESE LINES!
% Email me at: awood [at] jjay.cuny.edu
% or at: awood [at] gradcenter.cuny.edu


\documentclass{beamer}

\usepackage{tikz}
\usetikzlibrary{calc}

\usepackage{forest}
\usepackage{verbatim}
\usepackage{color}


\setbeamertemplate{footline}[frame number]
\setbeamertemplate{navigation symbols}{} 

\newtheorem{thm}{Theorem}[section]
\newtheorem{lem}{Lemma}
\newtheorem{cl}{Claim}
\newtheorem{cor}{Corollary}[section]
\newtheorem{conj}{Conjecture}
\newtheorem{quest}{Question}
\newtheorem{defn}{Definition}[section]
\newtheorem{obs}{Observation}[section]
\newtheorem{exam}{Example}

\newcommand{\im}{\operatorname{im}}
\newcommand{\id}{\operatorname{id}}
\newcommand{\interior}{\operatorname{int}}
\newcommand{\bdry}{\operatorname{bdry}}
\newcommand{\<}{\langle}
\renewcommand{\>}{\rangle}
\newcommand{\Gab}{(G_\phi)^{ab}} 
\newcommand{\phibar}{\bar{\phi}}
\newcommand{\Z}{\mathbb{Z}}
\newcommand{\N}{\mathbb{N}}
\newcommand{\Q}{\mathbb{Q}}
\newcommand{\R}{\mathbb{R}}
\newcommand{\C}{\mathbb{C}}
\newcommand{\A}{\mathcal{A}}
\newcommand{\OO}{\mathcal{O}}
\newcommand{\UU}{\mathcal{U}}
\newcommand{\power}{2^{\{P_1, \cdots , P_n\}}}
\newcommand{\bp}{\begin{problem}}
\newcommand{\ep}{\end{problem}}
\newcommand{\ba}{\begin{answer}}
\newcommand{\ea}{\end{answer}}
\newcommand{\ds}{\displaystyle}
\newcommand{\ben}{\renewcommand{\theenumi}{\alph{enumi}}
\renewcommand{\labelenumi}{(\theenumi)}\begin{enumerate}}
\newcommand{\een}{\end{enumerate}}
\newcommand{\Hess}{\operatorname{Hessian}}
\newcommand{\Aut}{\mathrm{Aut}}
\newcommand{\Inn}{\mathrm{Inn}}
\newcommand{\Out}{\mathrm{Out}}
\newcommand{\End}{\mathrm{End}}


\mode<presentation>
{
%  \usetheme{default}
  \setbeamercovered{invisible}
}


\usepackage[english]{babel}
\usepackage[latin1]{inputenc}
\usepackage{times}
\usepackage[T1]{fontenc}
\usepackage{stmaryrd}

%\usetheme{default}
%\usetheme{AnnArbor}
%\usetheme{Antibes}
%\usetheme{Bergen}
%\usetheme{Berkeley}
%\usetheme{Berlin}
%\usetheme{Boadilla}
%\usetheme{CambridgeUS}
%\usetheme{Copenhagen}
%\usetheme{Darmstadt}
%\usetheme{Dresden}
%\usetheme{Frankfurt}
%\usetheme{Goettingen}
%\usetheme{Hannover}
%\usetheme{Ilmenau}
%\usetheme{JuanLesPins}
%\usetheme{Luebeck}
%\usetheme{Madrid}
%\usetheme{Malmoe}
%\usetheme{Marburg}
%\usetheme{Montpellier}
%\usetheme{PaloAlto}
%\usetheme{Pittsburgh}
%\usetheme{Rochester}
\usetheme{Singapore}
%\usetheme{Szeged}
%\usetheme{Warsaw}

%\usecolortheme{default}
%\usecolortheme{albatross}
\usecolortheme{beaver}
%\usecolortheme{beetle}
%\usecolortheme{crane}
%\usecolortheme{dolphin}
%\usecolortheme{dove} % grey, white, yellow
%\usecolortheme{fly} %grey, yellow
%\usecolortheme{lily} %white, yellow, blue
%\usecolortheme{orchid}
%\usecolortheme{rose}
%\usecolortheme{seagull}
%\usecolortheme{seahorse}
%\usecolortheme{whale}
%\usecolortheme{wolverine}

% Title page

\title[CSCI360]{Basic Statistics}

\author
{Lecture notes of Alexander Wood \\ CSCI 360 Cryptography and Cryptanalysis \\ \scriptsize \href{mailto:awood@jjay.cuny.edu}{awood@jjay.cuny.edu}}
\institute[JJay]{John Jay College of Criminal Justice}  

\date{}

\begin{document}

% Remove 'figure' text from figure captions 
\setbeamertemplate{caption}{\raggedright\insertcaption\par}

\begin{frame}
  \titlepage
\end{frame}


\begin{frame}
\frametitle{Statistical Profiles}

Statistical profiles can mean many things. For instance, last class we created a statistical profile of single letter distributions in a text document. 
\end{frame}

\begin{frame}
\frametitle{Probability Distributions}

When we build statistical profiles we will be creating \textbf{probability distributions}. A probability distribution is a set of possible \textbf{outcomes}, $\Omega = \{x_1, x_2, \dots, x_n\}$, together with probabilities $P(X = x_1), P(X = x_2), \cdots, P(X = x_n)$ which represent the \textbf{likelihoods} of the respective outcomes occurring.\newline

The symbol $X$ above denotes the (unknown) actual outcome, and $P(X=x_i)$ denotes the probability that the \textbf{outcome} is $x_i$. 
\end{frame}

\begin{frame}
\frametitle{Example 1: D=Coin Flip}

Lets build a probability distribution for a single coin flip. The possible outcomes are heads or tails, denoted $\Omega = \{H,T\}$. The probability of each is $1/2$.
\[
P(X = H) = P(X = T) = 1/2
\]
\end{frame}

\begin{frame}
\frametitle{Example 2: Single Roll of 6-Sided Die}

There are six outcomes: $\Omega = \{1, 2, 3, 4, 5, 6\}$, each with equal likelihood.
\[
P(X = 1) = P(X = 2) = P(X = 3) 
\]
\[
= P(X = 4) = P(X = 5) = P(X = 6) = 1/6
\]
\end{frame}


\begin{frame}
\frametitle{Example 3: Two Rolls of a Fair-Sided Die}

Now lets look at an example where each outcome does not have the same likelihood. The sum of two rolls of a fair-sided die can take values $\Omega = \{2,3,4,5,6,7,8,9,10,11,12\}$\newline
\end{frame}


\begin{frame}
\frametitle{Example 3 Continued}

Below is the probability $P$ that the roll of two fair six sided dice will add up to $\Omega$.
\begin{center}
\begin{tabular}{|c|c|}
\hline 
$\Omega$ &	$P$ \\
\hline
2 &	 3\% \\
3 &	 6\% \\
4	&  8\% \\
5 &	 11\% \\
6 &	 14\% \\
7 & 	 17\% \\
8 &	 14\% \\
9 &	 11\% \\
10	 & 8\% \\
11	& 6\% \\ 
12 &	 3\% \\
\hline 
\end{tabular} 
\end{center}
\end{frame}

\begin{frame}
\frametitle{References}

\begin{itemize}
\item The random module: \url{https://docs.python.org/3/library/random.html}
\end{itemize}
\end{frame}
\end{document}


