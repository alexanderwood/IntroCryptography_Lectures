% These are the lecture notes for my CSCI360 course SPRING 2017
% at John Jay College of Criminal Justice. They are based largely on
% Schneier's Applied Cryptography.

% Feel free to edit these slides and use them for your own courses.
% HOWEVER DO NOT REMOVE THESE LINES!
% Email me at: awood [at] jjay.cuny.edu
% or at: awood [at] gradcenter.cuny.edu


\documentclass{beamer}

\usepackage{tikz}
\usetikzlibrary{calc}

\usepackage{graphicx}
\usepackage{amsmath}
\usepackage{enumerate}
%\usepackage{ntheorem}
%\usepackage[margin=1in]{geometry}
\usepackage{amssymb}
\usepackage{pdfpages}
%\usepackage{mathtools}
\usepackage{hyperref}
\usepackage{tikz}
\usepackage{booktabs}

\usepackage{forest}
\usepackage{verbatim}
\usepackage{color}
\usepackage{caption}


\usetikzlibrary{matrix,shapes,arrows,positioning,chains, calc}



\setbeamertemplate{footline}[frame number]
\setbeamertemplate{navigation symbols}{} 

\newtheorem{thm}{Theorem}[section]
\newtheorem{lem}{Lemma}
\newtheorem{cl}{Claim}
\newtheorem{cor}{Corollary}[section]
\newtheorem{conj}{Conjecture}
\newtheorem{quest}{Question}
\newtheorem{defn}{Definition}[section]
\newtheorem{obs}{Observation}[section]
\newtheorem{exam}{Example}

\newcommand{\im}{\operatorname{im}}
\newcommand{\id}{\operatorname{id}}
\newcommand{\interior}{\operatorname{int}}
\newcommand{\bdry}{\operatorname{bdry}}
\newcommand{\<}{\langle}
\renewcommand{\>}{\rangle}
\newcommand{\Gab}{(G_\phi)^{ab}} 
\newcommand{\phibar}{\bar{\phi}}
\newcommand{\Z}{\mathbb{Z}}
\newcommand{\N}{\mathbb{N}}
\newcommand{\Q}{\mathbb{Q}}
\newcommand{\R}{\mathbb{R}}
\newcommand{\C}{\mathbb{C}}
\newcommand{\A}{\mathcal{A}}
\newcommand{\OO}{\mathcal{O}}
\newcommand{\UU}{\mathcal{U}}
\newcommand{\power}{2^{\{P_1, \cdots , P_n\}}}
\newcommand{\bp}{\begin{problem}}
\newcommand{\ep}{\end{problem}}
\newcommand{\ba}{\begin{answer}}
\newcommand{\ea}{\end{answer}}
\newcommand{\ds}{\displaystyle}
\newcommand{\ben}{\renewcommand{\theenumi}{\alph{enumi}}
\renewcommand{\labelenumi}{(\theenumi)}\begin{enumerate}}
\newcommand{\een}{\end{enumerate}}
\newcommand{\Hess}{\operatorname{Hessian}}
\newcommand{\Aut}{\mathrm{Aut}}
\newcommand{\Inn}{\mathrm{Inn}}
\newcommand{\Out}{\mathrm{Out}}
\newcommand{\End}{\mathrm{End}}
\newcommand{\inv}{^{-1}}


\mode<presentation>
{
%  \usetheme{default}
  \setbeamercovered{invisible}
}


\usepackage[english]{babel}
\usepackage[latin1]{inputenc}
\usepackage{times}
\usepackage[T1]{fontenc}
\usepackage{stmaryrd}

%\usetheme{default}
%\usetheme{AnnArbor}
%\usetheme{Antibes}
%\usetheme{Bergen}
%\usetheme{Berkeley}
%\usetheme{Berlin}
%\usetheme{Boadilla}
%\usetheme{CambridgeUS}
%\usetheme{Copenhagen}
%\usetheme{Darmstadt}
%\usetheme{Dresden}
%\usetheme{Frankfurt}
%\usetheme{Goettingen}
%\usetheme{Hannover}
%\usetheme{Ilmenau}
%\usetheme{JuanLesPins}
%\usetheme{Luebeck}
%\usetheme{Madrid}
%\usetheme{Malmoe}
%\usetheme{Marburg}
%\usetheme{Montpellier}
%\usetheme{PaloAlto}
%\usetheme{Pittsburgh}
%\usetheme{Rochester}
\usetheme{Singapore}
%\usetheme{Szeged}
%\usetheme{Warsaw}

%\usecolortheme{default}
%\usecolortheme{albatross}
\usecolortheme{beaver}
%\usecolortheme{beetle}
%\usecolortheme{crane}
%\usecolortheme{dolphin}
%\usecolortheme{dove} % grey, white, yellow
%\usecolortheme{fly} %grey, yellow
%\usecolortheme{lily} %white, yellow, blue
%\usecolortheme{orchid}
%\usecolortheme{rose}
%\usecolortheme{seagull}
%\usecolortheme{seahorse}
%\usecolortheme{whale}
%\usecolortheme{wolverine}

% Title page

\title[Primes]{Diffie-Hellman, Implementation}
\subtitle{Based on  \emph{Cryptography Engineering} by Schneier, Ferguson, Kohno, Chapter 11}

\author{Lecture notes of Alexander Wood \\ \scriptsize \href{mailto:awood@jjay.cuny.edu}{awood@jjay.cuny.edu}}
\institute[JJay]{John Jay College of Criminal Justice}  

\date{}

\begin{document}

% Remove 'figure' text from figure captions 
\setbeamertemplate{caption}{\raggedright\insertcaption\par}

\begin{frame}
  \titlepage
\end{frame}

\begin{frame}
\frametitle{Diffie-Hellman Key Exchange}
Public information: Large prime $p$, a generator $g\in_R\Z_p^*$.
\[
\begin{array}{@{}l@{}c@{}l@{}}
\toprule
\textbf{Alice} && \textbf{Bob} \\

x \in_R \Z_p^*&& y \in_R \Z_p^*\\

& \xrightarrow{\textstyle  \text{   }\text{ }  g^x\pmod p  \text{   }\text{ } } \\

& \xleftarrow {\textstyle \text{   }\text{ }  g^y \pmod p \text{ }  \text{ } } \\

\kappa = g^{xy} \pmod p && \kappa = g^{xy} \pmod p \\
\bottomrule
\end{array}
\]
\end{frame}


\begin{frame}
\frametitle{Pitfall: $g$ Is Not Primitive}

What if $g$ is not a primitive element in $\Z_p^*$? Recall this means that the subgroup generated by $g$ is \emph{not} the whole group! \newline
\end{frame}

\begin{frame}
\frametitle{Pitfall: $g$ Is Not Primitive}

Say $g$ is a non-primitive element of order $1,000,000$. This means that the set $\{1, g, g^2, \dots, g^{q-1}\}$ contains exactly $1,000,000$ elements. \newline

This is effectively small enough for Eve to carry out a brute-force search for the generator.
\end{frame}

\begin{frame}
\frametitle{Pitfall: $g$ Is Not Primitive}

To protect against this we see it is important that Alice and Bob are able to verify that $p$ and $g$ were both chosen properly.\newline

Namely, they need to check that $p$ is a ``large'' prime and that $g$ is primitive modulo $p$. 
\end{frame}


\begin{frame}
\frametitle{Pitfalls: Eve Intercepts!}

What if Eve intercepts the values $g^x$ and $g^y$ sent by Bob and Alice and replaces them with $1$?
\[
\begin{array}{lcccccr}
\toprule
\textbf{Alice} &&\textbf{Eve}&& \textbf{Bob}  \\

x \in_R \Z_p^*&&&& y \in_R \Z_p^* \\

& \xrightarrow{ g^x} &&\xrightarrow{1}&\\

&\xleftarrow{g^y}&& \xleftarrow { g^y }  \\

\kappa = g^{xy} &&&& \kappa = 1  \\
\bottomrule
\end{array}
\]
The protocol \emph{looks} like it completed correctly to Alice And Bob -- except now Eve knows the key Bob is using!
\end{frame}

\begin{frame}
\frametitle{Pitfalls: Eve Intercepts!}

This scenario is easy enough to protect against -- just have Bob check that his key is not $1$. \newline

There are more sophisticated methods Eve can use, however, which are less easily detectable by Alice or Bob. 
\end{frame}


\begin{frame}
\frametitle{Small Order Element Replacement Attack}

Instead of replacing $g^x$ with $1$, now Eve will replace it with $h\in\Z_p^*$ where $h$ has small order. 
\end{frame}



\begin{frame}
\frametitle{Small Order Element Replacement Attack}

Recall that the \textbf{order} of an element $h$ modulo $p$ is the smallest positive value $q$ which satisfies 
\[
h^q = 1 \pmod p.
\]
This means that $h$ \textbf{generates}  a $q$-element subgroup of $\Z_p^*$,
\[
\{1, h, h^2, \dots, h^{q-1}\}
\]
\end{frame}


\begin{frame}
\frametitle{Small Order Element Replacement Attack}

So when we say that Eve replaces $g^x$ with $h$ such that $h$ has small order, we deduce that the subgroup generated by $h$ is small! \newline

How can Eve use this to her advantage?
\end{frame}


\begin{frame}
\frametitle{Small Order Element Replacement Attack}

Let's look at this attack in more detail. 
\[
\begin{array}{lcccccr}
\toprule
\textbf{Alice} &&\textbf{Eve}&& \textbf{Bob}  \\

x \in_R \Z_p^*&&&& y \in_R \Z_p^* \\

& \xrightarrow{ g^x} &&\xrightarrow{h}&\\

&\xleftarrow{g^y}&& \xleftarrow { g^y }  \\

\kappa = g^{xy} &&&& \kappa = h^y  \\
\bottomrule
\end{array}
\]
Eve knows $h$, but not $y$. How can Eve determine the value of $h^y$?
\end{frame}


\begin{frame}
\frametitle{Small Order Element Replacement Attack}

Eve knows $h$, and knows that Bob's key will be $h^y$ for some $y$. All Eve has to do to decrypt a message Bob sends is try every possible value which $h^y$ can take -- which is not very many, since the subgroup generated by $h$ is small.  \newline

How can Alice and Bob protect themselves against \emph{this} attack?
\end{frame}



\begin{frame}
\frametitle{Small Order Element Replacement Attack}

To protect against this attack, Alice and Bob must verify that they numbers they receive from one another do not generate small subgroups.
\end{frame}


\begin{frame}
\frametitle{Subgroups}

We should notice a common theme -- Alice and Bob need to be able to deduce information about subgroups modulo $\Z_p^*$.
\end{frame}


\begin{frame}
\frametitle{Subgroups}

All multiplicative subgroups modulo a prime $p$ can be generated from a single element. This means that each subgroup can be written as
\[
\{ 1, h, h^2, h^2, \dots, h^{q-1}\}
\]
for some $h\in\Z_p^*$, where $q$ is the order of $h$. 
\end{frame}


\begin{frame}
\frametitle{Subgroups}

The order $q$ of any subgroup of $\Z_p^*$ must be a divisor of $p-1$. \newline

Conversely, for every divisor $q$ of $p-1$, there is a subgroup of $\Z_p^*$ of order $q$. \newline

\emph{How can we use this information to avoid small subgroups modulo $p$?}
\end{frame}


\begin{frame}
\frametitle{Subgroups}
\emph{How can we use this information to avoid small subgroups modulo $p$?}\newline

We must avoid having small divisors of $p-1$. Why is this a problem?
\end{frame}

\begin{frame}
\frametitle{Subgroups}

We want $p-1$ to have no small divisors. However every large prime is odd, meaning $p-1$ is always even. Hence $2$ is always a small divisor of $p-1$.\newline

We work around this by asserting that $p-1$ have no small factors besides $2$.
\end{frame}


\begin{frame}
\frametitle{Safe Primes}

We wish to define a {safe prime} for $p$ which we may use in the protocol. A \textbf{safe prime} $p$ is a ``large enough'' prime of the form
\[
p = 2q + 1
\]
where $q$ is also prime. 
\end{frame}


\begin{frame}
\frametitle{Safe Primes}

For a safe prime $p$ we can determine all subgroups of $\Z_p^*$. In fact, the subgroups are given by
\begin{enumerate}[(1)]
\item The trivial subgroup, $\{1\}$
\item The subgroup of order $2$, given by $\{1, p-1\}$
\item The subgroup of size $q$.
\item The full group of size $2q$.
\end{enumerate}
\end{frame}

\begin{frame}
\frametitle{Safe Primes}

It is easy to avoid subgroups (a) and (b). We will wish to use subgroups of form (c), which are of order $q$. \newline

We need a way to distinguish between subgroups of order $q$ and subgroups of order $2q$.
\end{frame}


\begin{frame}
\frametitle{Squares Modulo $p$}

Consider the set of all numbers $x$ which are \textbf{square} modulo $p$. This means that there exists a $y\in\Z_p^*$ such that 
\[
x = y^2 \pmod{p}
\]
In other words, $x$ is the square of some number $y$. 
\end{frame}


\begin{frame}
\frametitle{Squares Modulo $p$}

Exactly half of the numbers modulo $p$ are squares, half are not squares. \newline

A generator of the entire group is a non-square. Why?	
\end{frame}

\begin{frame}
\frametitle{Squares Modulo $p$}

A square element modulo $p$ can never generate a non-square. Observer, if $x=y^2\pmod p$, then
\[
x^j = (y^j)^2 \pmod p
\]
for all integers $j$. 
\end{frame}

\begin{frame}
\frametitle{Squares Modulo $p$}

Therefore, \textbf{any generator of the entire group is a non-square}.\newline

Furthermore, \textbf{any generator of the subgroup of order $q$ is a square}.
\end{frame}


\begin{frame}
\frametitle{Legendre Symbol}

A mathematical function called the \textbf{Legendre symbol} is able to calculate whether a number is square modulo $p$ efficiently. \newline

Alice and Bob should use an efficient Legendre symbol algorithm to verify that $g$ is a non-square. 
\end{frame}


\begin{frame}
\frametitle{Legendre Symbol}

If $x$ is odd, then $g^x$ is a non-square. If $x$ is even then $g^x$ is square. Therefore, Eve is able to deduce whether Alice's private value $x$ is even or odd. \newline

To avoid this problem, use only squares modulo $p$. This is the subgroup of order $q$. Since $q$ is prime, there are no further subgroups for us to worry about!
\end{frame}


\begin{frame}
\frametitle{Finding Safe Primes}

The following is an algorithm for using a safe prime.\newline

\begin{enumerate}[(1)]
\item Choose $(p,q)$ such that $p=2q+1$, and $p$ and $q$ are prime.
\item Choose a random $\alpha$ in the range $2,\dots, p-2$. 
\item Set $g=\alpha^2 \pmod p$. 
\item Check that $g\ne 1$ and $g \ne p-1$. If $g$ equals one of these values, repeat from step $(2)$.
\item Output parameter $(p, q, g)$, suitable for Diffie-Hellman.
\end{enumerate}
\end{frame}


\begin{frame}
\frametitle{Using Smaller Subgroups}

While the safe prime approach works, it is inefficient. For an $n$-bit prime $p$, the value $q$ is $n-1$ bits long and hence all exponents are $n-1$ bits long. Exponentiation takes, on average, $3n/2$ multiplications modulo $p$. 
\end{frame}


\begin{frame}
\frametitle{Using Smaller Subgroups}

We can work around this by using smaller subgroups. However, we must be careful when we do this to avoid the security pitfalls mentioned earlier.
\end{frame}


\begin{frame}[fragile]
\frametitle{Using Smaller Subgroups}

Following is an algorithm for generating the security parameter using a smaller subgroup. 

\begin{enumerate}[(1)]
\item Choose a \verb|256|-bit prime $q$.
\item Find a prime $p = Nq+1$ for a large arbitrary (even) value $N$. ($p$ should be thousands of bits long.)
\item Find an element of order $q$ as follows:
	\begin{itemize}
	\item $\alpha\in_R\Z_p^*$, set $g\leftarrow \alpha^N$
	\item Verify that $g\ne 1$ and $g^q = 1$. Repeat until this is satisfied. 
	\end{itemize}
\item Output parameter $(p, g, q)$.
\end{enumerate}
\end{frame}


\begin{frame}
\frametitle{Using Smaller Subgroups}

The values Bob and Alice exchange are in the subgroup generated by $g$, which is now of order $q$. To verify that Eve is not substituting a different value of $g$ Alice and Bob should now check that the number they received, say $r$, is in fact in the subgroup generated by $g$. This reduces to verifying the following:
\begin{itemize}
\item $r\ne 1$, and
\item $r^q = 1 \pmod p$, and
\item $1 < r < p$
\end{itemize}
\end{frame}

\begin{frame}
\frametitle{Using Smaller Subgroups}

Why is this faster than the safe primes case? Well, since $r^q=1$ for all $r$ in the subgroup generated by $g$, we have that
\[
r^e = r^{(e\mod {q})} \pmod p
\]
which is much faster to compute!
\end{frame}

\begin{frame}[fragile]
\frametitle{Efficiency of Using Smaller Subgroups}

The prime $p$ is at least \verb|2000| bits long! With safe primes it takes $3000$ multiplications to compute $g^x$; in the smaller subgroups case, it takes only $384$ multiplications. Wow!
\end{frame}


\begin{frame}[fragile]
\frametitle{Size of $p$}

In symmetric-key protocols we looked at previously, key sizes were able to be much smaller. Public-key sizes are much larger than symmetric-key sizes, and public-key protocols tend to run more slowly. Key should be on the order of thousands of bits; \verb|2048| is an absolute minimum, but it is recommended to go as close to \verb|4096| or higher that your system can handle. 
\end{frame}


\begin{frame}
\frametitle{References}

\begin{itemize}
\item \emph{Cryptography Engineering} by Schneier, Ferguson, Kohno, Chapter 11
\end{itemize}
\end{frame}
\end{document}


