% These are the lecture notes for my CSCI360 course SPRING 2017
% at John Jay College of Criminal Justice. They are based largely on
% Schneier's Applied Cryptography.

% Feel free to edit these slides and use them for your own courses.
% HOWEVER DO NOT REMOVE THESE LINES!
% Email me at: awood [at] jjay.cuny.edu
% or at: awood [at] gradcenter.cuny.edu


\documentclass{beamer}

\usepackage{tikz}
\usetikzlibrary{calc}

\usepackage{forest}
\usepackage{verbatim}
\usepackage{color}
\usepackage{amsmath}
\usepackage{graphicx}
\usepackage{caption}



\setbeamertemplate{footline}[frame number]
\setbeamertemplate{navigation symbols}{} 

\newtheorem{thm}{Theorem}[section]
\newtheorem{lem}{Lemma}
\newtheorem{cl}{Claim}
\newtheorem{cor}{Corollary}[section]
\newtheorem{conj}{Conjecture}
\newtheorem{quest}{Question}
\newtheorem{defn}{Definition}[section]
\newtheorem{obs}{Observation}[section]
\newtheorem{exam}{Example}

\newcommand{\im}{\operatorname{im}}
\newcommand{\id}{\operatorname{id}}
\newcommand{\interior}{\operatorname{int}}
\newcommand{\bdry}{\operatorname{bdry}}
\newcommand{\<}{\langle}
\renewcommand{\>}{\rangle}
\newcommand{\Gab}{(G_\phi)^{ab}} 
\newcommand{\phibar}{\bar{\phi}}
\newcommand{\Z}{\mathbb{Z}}
\newcommand{\N}{\mathbb{N}}
\newcommand{\Q}{\mathbb{Q}}
\newcommand{\R}{\mathbb{R}}
\newcommand{\C}{\mathbb{C}}
\newcommand{\A}{\mathcal{A}}
\newcommand{\OO}{\mathcal{O}}
\newcommand{\UU}{\mathcal{U}}
\newcommand{\power}{2^{\{P_1, \cdots , P_n\}}}
\newcommand{\bp}{\begin{problem}}
\newcommand{\ep}{\end{problem}}
\newcommand{\ba}{\begin{answer}}
\newcommand{\ea}{\end{answer}}
\newcommand{\ds}{\displaystyle}
\newcommand{\ben}{\renewcommand{\theenumi}{\alph{enumi}}
\renewcommand{\labelenumi}{(\theenumi)}\begin{enumerate}}
\newcommand{\een}{\end{enumerate}}
\newcommand{\Hess}{\operatorname{Hessian}}
\newcommand{\Aut}{\mathrm{Aut}}
\newcommand{\Inn}{\mathrm{Inn}}
\newcommand{\Out}{\mathrm{Out}}
\newcommand{\End}{\mathrm{End}}


\mode<presentation>
{
%  \usetheme{default}
  \setbeamercovered{invisible}
}


\usepackage[english]{babel}
\usepackage[latin1]{inputenc}
\usepackage{times}
\usepackage[T1]{fontenc}
\usepackage{stmaryrd}

%\usetheme{default}
%\usetheme{AnnArbor}
%\usetheme{Antibes}
%\usetheme{Bergen}
%\usetheme{Berkeley}
%\usetheme{Berlin}
%\usetheme{Boadilla}
%\usetheme{CambridgeUS}
%\usetheme{Copenhagen}
%\usetheme{Darmstadt}
%\usetheme{Dresden}
%\usetheme{Frankfurt}
%\usetheme{Goettingen}
%\usetheme{Hannover}
%\usetheme{Ilmenau}
%\usetheme{JuanLesPins}
%\usetheme{Luebeck}
%\usetheme{Madrid}
%\usetheme{Malmoe}
%\usetheme{Marburg}
%\usetheme{Montpellier}
%\usetheme{PaloAlto}
%\usetheme{Pittsburgh}
%\usetheme{Rochester}
\usetheme{Singapore}
%\usetheme{Szeged}
%\usetheme{Warsaw}

%\usecolortheme{default}
%\usecolortheme{albatross}
\usecolortheme{beaver}
%\usecolortheme{beetle}
%\usecolortheme{crane}
%\usecolortheme{dolphin}
%\usecolortheme{dove} % grey, white, yellow
%\usecolortheme{fly} %grey, yellow
%\usecolortheme{lily} %white, yellow, blue
%\usecolortheme{orchid}
%\usecolortheme{rose}
%\usecolortheme{seagull}
%\usecolortheme{seahorse}
%\usecolortheme{whale}
%\usecolortheme{wolverine}

% Title page

\title[Randomness]{Generating Randomness}
\subtitle{Based on  \emph{Cryptography Engineering} by Schneier, Ferguson, Kohno, Chapter 9}

\author
{Lecture notes of Alexander Wood \\ \scriptsize \href{mailto:awood@jjay.cuny.edu}{awood@jjay.cuny.edu}}
\institute[JJay]{John Jay College of Criminal Justice}  

\date{}

\begin{document}

% Remove 'figure' text from figure captions 
\setbeamertemplate{caption}{\raggedright\insertcaption\par}

\begin{frame}
  \titlepage
\end{frame}


\begin{frame}
\frametitle{True Randomness}

What is \textbf{true randomness}? 
\end{frame}

\begin{frame}
\frametitle{True Randomness}

Avoiding too technical of a discussion, true randomness should be truly \textbf{unpredictable}. If we are witnessing a random stream of numbers, knowing the values up to one point will tell us absolutely nothing about the numbers to come. 
\end{frame}

\begin{frame}
\frametitle{Generating Randomness}

How can we generate randomness, which is \textbf{nondeterministic}, using computers, why are by definition \textbf{deterministic}?
\end{frame}

\begin{frame}
\frametitle{Pseudo-Random Number Generators}

We will discuss \textbf{pseudo-random number generators (PRNGs)} which generate numbers that are not \emph{truly} random, but rather \textbf{computationally indistinguishable} from truly random numbers.
\end{frame}

\begin{frame}
\frametitle{Vid eo}

Let's begin by watching this excellent video by Khan Academy for an overview.\newline

\url{https://www.youtube.com/watch?v=GtOt7EBNEwQ}
\end{frame}

\begin{frame}[fragile]
\frametitle{Entropy}

\textbf{Entropy} provides a measure for randomness. A random $64$-bit string can take on $2^{64}$ possible values and hence has $64$ bits of entropy. \newline
\end{frame}

\begin{frame}[fragile]
\frametitle{Entropy (Example)}
Suppose instead that a $64$-bit string contains exactly $60$ \verb|0|'s and $4$ \verb|1|'s. This means there are ${64 \choose 4} = 635376$ possible values this bit string can take on. \newline

Since $log_2(635375) \approx 19.3$, this means there are approximately $2^{19.3}$ possible values -- yielding an entropy of $19.3$ bits for our $64$ bit string.
\end{frame}


\begin{frame}[fragile]
\frametitle{Exercise: Entropy}

\begin{enumerate}[(a)]
\item Suppose you have a \verb|100|-bit string that contains exactly $79$ \verb|1|'s. How many bits of entropy does this string contain? 
\item What if it contains exactly $99$ \verb|1|'s? 
\item Exaclty $50$ \verb|1|'s? 
\end{enumerate}
\end{frame}

\begin{frame}[fragile]
\frametitle{Exercise: Entropy}

\emph{Answer (a)}: With exactly $79$ \verb|1|'s it contains
\[
\log_2\left({100 \choose 79}\right) \approx 70.8
\]
bits of entropy.
\end{frame}

\begin{frame}[fragile]
\frametitle{Exercise: Entropy}
 \emph{Answer (b) }: With exactly $99$ \verb|1|'s it contains
\[
\log_2\left({100 \choose 99}\right) \approx 6.6
\]
bits of entropy.
\end{frame}


\begin{frame}[fragile]
\frametitle{Exercise: Entropy}
 \emph{Answer (b) }: With exactly $50$ \verb|1|'s it contains
\[
\log_2\left({100 \choose 50}\right) \approx 93.3
\]
bits of entropy.
\end{frame}

\begin{frame}
\frametitle{Entropy}

\emph{Why is the entropy for 99 ones so much smaller than the entropy for 50 ones?} \newline \pause

The more we know about a string, the less entropy it has. 
\end{frame}


\begin{frame}
\frametitle{Formal Entropy Definition}

Given a probability distribution $P$ over a range of values $X$, the entropy $H$ is given by
\[
H(X) := - \sum_{x} P(X = x) \log_2 P(X = x).
\]
With the \emph{uniform distribution} over bit strings as above, the calculation reduces to what we performed on the previous slides. (Do not worry about this definition, I only include it so that you've seen it.)
\end{frame}

\begin{frame}
\frametitle{True Randomness}

Why do we not use truly random data for cryptographic applications? Discuss.
\end{frame}

\begin{frame}
\frametitle{Generating Randomness}

Possible methods of generating randomness are the timing of keystrokes or having the user input random data themselves. However, these methods will not be very reliable -- they would have too low of entropy. These processes are less random than you would think.
\end{frame}

\begin{frame}
\frametitle{Generating Pseudorandomness}

Instead we try and generate pseudorandomness. Pseudoranomness is generated from a random seed by some deterministic algorithm called a \textbf{pseudorandom number generator (PRNG)}. \newline

Always assume that an adversary knows your PRNG! What you want to keep secret is your \textbf{key}.
\end{frame}

\begin{frame}
\frametitle{PRNG Security}

To measure the security of a PRNG we want to ask the following question: Given some pseudorandom outputs, it it possible for an attacker to predict future or past pseudorandom bits? \newline

For a cryptographically secure PRNG the answer must be a resounding \textbf{no!} \newline

This means that \textbf{an attacker cannot distinguish between a truly random sequence and your pseudo random sequence}.
\end{frame}

\begin{frame}[fragile]
\frametitle{Selecting a PRNG}

Be very careful when selecting a PRNG and make sure that it is documented as \textbf{cryptographically} secure. \newline

For instance, \verb|random.randint(a,b)| in Python's \verb|random| module generates a pseudo-random number between $a$ and $b$ -- however, it is not a cryptographically secure PRNG. The random module should be used for modeling and testing, never for true cryptographic applications. \newline
\end{frame}

\begin{frame}[fragile]
\frametitle{Selecting a PRNG}

In Python you can use the \verb|os| module to generate random values. \url{https://docs.python.org/3/library/os.html}\newline

Python's \verb|secrets| module has some options. It uses random values generated by your os as well, but streamlines a lot of the work for you. \url{https://docs.python.org/3/library/secrets.html}
\end{frame}

\begin{frame}
\frametitle{PRNGs}

NIST provides a reference for PRNGs: \url{http://csrc.nist.gov/groups/ST/toolkit/random_number.html}\newline

Some PRNG algorithms that are practical for cryptographic use are Yarrow, ChaCha20, Fortuna, ISAAC, and more. 
\end{frame}

\begin{frame}
\frametitle{PRNG: Fortuna}

Fortuna (named for the Roman goddess of chance) was designed by Schneier and Ferguson and published in 2003. It is based upon block ciphers. It has three parts:
\begin{enumerate}[(1)]
\item The \textbf{generator}, which takes a fixed size seed as input and generates an arbitrary amount of pseudo random data.
\item The \textbf{accumulator}, which collects entropy for various sources in order to re-seed the generator.
\item The \textbf{seed file control}, which ensures that random data is being generated regardless of when the computer was booted up.
\end{enumerate}
\end{frame}


\begin{frame}
\frametitle{PRNG: Fortuna}

To learn how Fortuna operates in detail, read 9.3 through 9.6 of \emph{Cryptography Engineering} by Schneier, Ferguson, Kohno.
\end{frame}


\begin{frame}
\frametitle{References}

\begin{itemize}
\item \emph{Cryptography Engineering} by Schneier, Ferguson, Kohno, Chapter 9
\end{itemize}
\end{frame}
\end{document}


